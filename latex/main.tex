%% packages
\documentclass{article}
\usepackage[a4paper, left=2.0cm, right=2.0cm, top=3.5cm]{geometry}
\usepackage[ngerman]{babel}
\usepackage{graphicx}
\usepackage{multicol}
\usepackage{amssymb}
\usepackage{titlesec}
\usepackage{wrapfig}
\usepackage{blindtext}
\usepackage{lipsum}
\usepackage{caption}
\usepackage{listings}
\usepackage{nopageno}
\usepackage{authblk}
\usepackage{amsmath} % tons of math stuff
\usepackage{esdiff} % provides \diff
\usepackage{xcolor}
\usepackage{csquotes} % e.g. provides \enquote
\usepackage{siunitx} % units
\usepackage{xcolor} % colored text
\usepackage{subcaption}

\graphicspath{{plot/}}

% own commands
% \newcommand{\rarr}{$\to\,$} %A$\,\to\,$B
\newcommand{\defc}{black}
\newcommand{\colorT}[2][blue]{\color{#1}{#2}\color{\defc}}
\newcommand{\redq}{\color{red}(?)\color{\defc}}
\newcommand{\question}[1]{\colorT[purple]{\textbf{(#1)}}}
\newcommand{\todo}[1]{\colorT[red]{\textbf{(#1)}}}
\newcommand{\mr}{\mathrm}

%% document
\begin{document}

%\pagenumbering{gobble}
%\maketitle
%\tableofcontents
%\newpage
\pagenumbering{arabic}

\pagestyle{fancy}
%\fancyhead[R]{\thepage}
%\fancyhead[L]{\leftmark}

Für ein festes $\mu$ werden die (über alle Konfigurationen gemittelten) 3pt-Funktionen für alle 4 möglichen $\nu$ gegen die Zeit aufgetragen: Abb. \ref{fig:3pt}.
\begin{figure}
    \centering
    \begin{subfigure}{0.49\linewidth}
        \centering
        \includegraphics[width=\textwidth]{confs_mu0}
        \subcaption{$\mu=0$}
    \end{subfigure}
    \hfill
    \begin{subfigure}{0.49\textwidth}
        \centering
        \includegraphics[width=\textwidth]{confs_mu1}
        \subcaption{$\mu=1$}
    \end{subfigure}

    \begin{subfigure}{0.49\textwidth}
        \centering
        \includegraphics[width=\textwidth]{confs_mu2}
        \subcaption{$\mu=2$}
    \end{subfigure}
    \hfill
    \begin{subfigure}{0.49\textwidth}
        \centering
        \includegraphics[width=\textwidth]{confs_mu3}
        \subcaption{$\mu=3$}
    \end{subfigure}
    \caption{Plots der 3pt-Funktionen.}
    \label{fig:3pt}
\end{figure}

\end{document}
